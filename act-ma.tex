% Sample Marathi act text
\documentclass[12pt]{article}
\usepackage{fontspec}
\usepackage{fullpage}
\usepackage{lastpage}
\usepackage{polyglossia}

\usepackage{fancyhdr}
\setmainlanguage{sanskrit}
\setotherlanguages{english}


\usepackage{refcount}


\fancyhead{}
\fancyfoot[c]{p\themrpage}
\pagestyle{fancy}


\newfontfamily\devanagarifont[Script=Devanagari]{Lohit Devanagari} 

%%% Justify without hyphenation
\tolerance=1
\emergencystretch=\maxdimen
\hyphenpenalty=10000
\hbadness=10000



\makeatletter
% https://tex.stackexchange.com/questions/70414/how-to-get-devanagari-numerals-in-latex-or-xetex

\newcommand{\devanagarinumeral}[1]{%
  \devanagaridigits{\number\csname c@#1\endcsname}}

% https://tex.stackexchange.com/questions/366321/numbering-enumerate-environments-and-pages-with-devanagari-alphabet

\def\devanagari@alph#1{%
 \ifcase#1\or क\or ख\or ग\or घ\or ङ\or च\or छ\or ज\or झ\or ञ\or
  ट\or ठ\or ड\or ढ\or ण\or त\or थ\or द\or ध\or न\or प\or फ\or ब\or भ\or म\or
   य\or र\or ल\or व\or श\or ष\or स\or ह\else\@ctrerr\fi}
%%Imitating xgreek.sty and xepersian.sty
\let\@alph\devanagari@alph
\let\@Alph\devanagari@alph
\let\@roman\devanagari@alph
\let\@Roman\devanagari@alph

% Change appearance of enumerate at levels 2, 3, 4.
%\renewcommand{\theenumi}{\devanagarinumeral{enumi}}
\renewcommand{\labelenumii}{\theenumii.} % Instead of: (\theenumii)
\renewcommand{\labelenumiii}{(\theenumiii)} % Instead of: \theenumiii.
\renewcommand{\theenumiii}{\@arabic\c@enumiii} % Instead of: \@roman\c@enumiii
\renewcommand{\labelenumiv}{(\theenumiv)} % Instead of: \theenumiv.

\makeatother

% renew all representation of counters
 \renewcommand{\thesection}{\devanagarinumeral{section}}
 \renewcommand{\thesubsection}{\thesection.\devanagarinumeral{subsection}}
 \newcommand{\themrpage}{\devanagarinumeral{page}}
% \renewcommand{\thepage}{\devanagarinumeral{page}}

\begin{document}


\newcommand{\setin}[1]{%
  \clearpage
%  \newfontfamily\devanagarifont[Script=Devanagari]{#1}
  \fontspec{#1}[Script=Devanagari,
   %Language=Marathi,
    Script=Devanagari,
  Mapping=devanagarinumerals,
 % StylisticSet=2
  ]
  
  \section{\foreignlanguage{english}{#1}}
  
  \subsection{प्रारंभिक} 
  %\section{प्रारंभिक  \foreignlanguage{english}{#1}}

१	२	३ 	४       ५	६	७	८	९\\
श्र	त्क	द्व	द्य	द्द	द्म	त्त त्र	र्य	र्‍य	प्र	ट्र	ह्य	ह्म\\
  
\begin{enumerate}
\item \textbf{नियोजन} {\Large \bfseries समतीचा} अहवाल मिळाल्यानंतर, दोन
    महिन्यापेक्षा अधिक नसेल अशा
कालावधीत, योजनेच्या मसुद्यात त्याला योग्य वाटतील असे फेरबदल किंवा बदल करील. नियोजन
प्राधिकरण किंवा उक्त अधिकारी, जनतेच्या माहितीसाठी विकास योजनेच्या मसुद्यात करण्यात
आलेल्या फेरबद\-लांची किंवा बदलांची सूची राजपत्रात व दोनपेक्षा कमी नसतील अशा स्थानिक
वर्तमानपत्रांत प्रसिद्ध करील.

\item अशाप्रकारे फेरबदल करण्यात आलेला विकास योजनेचा मसुदा मंजुरीसाठी राज्य शासनाकडे
सादर करण्यापूर्वी किमान एक महिना अगोदर तो मसुदा राजपत्रात आणि विहित करण्यात येईल
अशा अन्य रीतीने प्रसिद्ध करण्यात येईल.

\item नियोजन प्राधिकरण किंवा, यथास्थिति, उक्त अधिकारी कलम 26 अन्वये 4[अशी
  योजना, तयार केली असल्याबद्दलची नोटीस राजपत्रामध्ये प्रसिद्ध करण्यात आल्याच्या
  दिनांकापासून] सहा महिन्यांच्या कालावधीत, कलम 28 च्या पोट-कलम (4) अन्वये,
  विकास योजनेच्या मसुद्यात करण्यात आलेल्या फेरबदलांच्या किंवा बदलांच्या सूचीसह,
  विकास योजनेचा मसुदा मंजुरीसाठी राज्य शासनास सादर करील; परंतु तो कालावधी कोणत्याही परिस्थितीत,---
  \begin{enumerate}
  \item अलिकडच्या जनगणनेच्या आकडेवारीनुसार, एक कोटी किंवा त्यापेक्षा अधिक लोकसंख्या
    असलेल्या महानगरपालिकांच्या बाबतीत, एकूण चोवीस महिने ;
  \item अलिकडच्या
    जनगणनेच्या आकडेवारीनुसार, दहा लाख किंवा त्यापेक्षा अधिक, परंतु एक कोटीपेक्षा
    कमी लोकसंख्या असलेल्या महानगर\-पालिकांच्या बाबतीत, एकूण बारा महिने ; आणि
    \item इतर कोणत्याही बाबतीत, एकूण सहा महिने, यापेक्षा अधिक असणार नाही.

    \item यामध्ये काहीही अंतर्भूत असले तरी, लोकहिताच्या दृष्टीने, अंतिम विकास
      योजनेच्या कोण\-त्याही भागात किंवा त्यामध्ये केलेल्या कोणत्याही प्रस्तावात,
      याबाबत राज्य शासनाची खात्री पटल्यास, अशा प्रकरणात, राज्य शासनास,
      नियोजित फेरबदलाच्या संबंधात
      \begin{enumerate}
      \item राज्य शासन विनिर्दिष्ट कालावधीनंतर, अशा सर्व आक्षेपांची व सूचनांची
        प्रत नियो\-जन प्राधिकरणाकडे त्याच्या विचारार्थ पाठवील. त्यानंतर, नियोजन
        प्राधिकरण, शासना\-कडून अशा आक्षेपांची व सूचनांची प्रत मिळाल्यापासून एक
        महिन्याच्या काला\-वधीत त्याचे म्हणणे शासनाकडे सादर करील.
\end{enumerate}

      
\end{enumerate}

\end{enumerate}

}

%\begin{raggedright}

Number of pages = \pageref{pg:lastpage}

12 in Devanagari: \devanagaridigits{12}

Page Count in devanagari \devanagaridigits{\number\getpagerefnumber{pg:lastpage}}

Page Count in devanagari \devanagaridigits{\number\getpagerefnumber{LastPage}}

%Page number: \devanagaridigits{\pageref{pg:lastpage}} 
%\devanagaridigits{\getrefbykeydefault{pg:lastpage}{page}{0}}


\setin{Shobhika}
\setin{CDAC-GISTSurekh}
\setin{Murty Sanskrit}

\setin{Lohit Devanagari}
\setin{Chandas}
\setin{Gargi}
\setin{Kalimati}
\setin{Nakula}
\setin{Samanata}
\setin{Samyak Devanagari}
\setin{SakalBharati}
\setin{Mukta Regular}
\setin{Sahadeva} % Old Devanagari
\setin{Sarai}

\label{pg:lastpage}

%\end{raggedright}
\end{document}

%%% Local Variables:
%%% mode: latex
%%% TeX-master: t
%%% End:
