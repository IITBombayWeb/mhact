%%% Sample New Act covering all the necessary features

\documentclass{mhact}
\usepackage{times}
\usepackage{layout}



\begin{document}

\actnamealias{2000: Mah. XVIII}{Maharashtra Act No. XVIII of 2000}
\shorttitle[Bombay Rents... (Amendment), 1999]{Bombay Rents, Hotel and
  Lodging House Rates Control, Bombay Land Requisition and Bombay
  Government Premises (Eviction) (Amendment) Act, 1999 }
\actedition{As modified up to 21 May 2013} \actedyear{2013}
\price{Rs 23}

\maketitle

\tableofcontents
\startact[For Statement of Objects and Reasons, see. Maharashtra
  Government Gazette, Part-V-A, pages 347 to 349.] 


\assent{%
  This Act received the assent of the President on 8 March 2000; assent
  was first published in the Maharashtra Government Gazette, Part-IV, on 
  10 March 2000.}

\longtitle{
  An Act to unify, consolidate and amend the law relating to the
  control of rent and repairs of certain premises and of eviction and
  for encouraging the construction of new houses by assuring a fair
  return on the investment by landlords and to provide for the matters
  connected with the purposes aforesaid.
}



\enactingtext{%
  WHEREAS it is expedient to unify, consolidate and amend the laws
  prevailing in the different parts of the State relating to the
  control of rents and repairs of certain premises and of eviction and
  for encouraging the construction of new houses by assuring a fair
  return and to provide for the matters connected with the purposes
  aforesaid; It is hereby enacted in the Fiftieth Year of the Republic
  of India as follows:--- }

\chapter{Preliminary}

\section{Short title, extent and commencement}

\begin{subsectionlist}
\item  This Act may be called the Maharashtra Rent Control Act,
  1999.
\item It shall extend to the whole of the State of Maharashtra. Short title, extent and
comencement.

\item It shall come into force on such date\footnote{ 31 March
    2000, vide, G. N., H \& SAD., No. MEA. 2000/CR-14/Bhanika, dated
    30 March 2000, published in Maharashtra Government
    Gazette, 2000, Part-IV-B, Extra No. 84, p. 297 } as the State
  Government may, by notification in the Official Gazette, appoint.
\end{subsectionlist}

\section{Application}
\begin{subsectionlist}
\item This Act shall, in the first instance, apply to premises let for
  the purposes of residence, education, business, trade or storage in
  the areas specified in Schedule I and Schedule II.
  \label{it:purpose}
\item Notwithstanding anything contained in sub-section
  \ref{it:purpose}, it shall
  also apply to the premises or as the case may be, houses let out in
  the areas to which the Bombay Rents, Hotel and Lodging House Rates
  Control Act, 1947\actref{Bom. LVII of 1947} or the Central
  Provinces and Berar Letting of Houses and Rent Control Order, 1949
  issued under the Central Provinces and Berar Regulation of Letting
  of Accomodation Act, 1946\actref{C.P. and Berar Act XI of 1946.}
  and Hyderabad Houses (Rent, Eviction and Lease) Control Act, 1954
  \actref{Hyd. Act No. XX of 1954} were extended and applied before
  the date of commencement of this Act and such premises or houses
  continue to be so let on that date in such areas which are specified
  in Schdedule I to this Act, notwithstanding that the area ceases to
  be of the description therein specified.

\item It shall also apply to the premises let for the purposes
  specified in sub-section \ref{it:purpose} in such of the cities or
  towns as specified in Schedule II.
\item Notwithstanding anything
  contained hereinabove, the State Government may, by notification in
  the Official Gazette, direct that---
  \begin{clause}
  \item this Act shall not apply to any of the areas specified in Schedule I or
    Schedule II or that it shall not apply to any one or all purposes specified in
    sub-section \ref{it:purpose};
  \item this Act shall apply to any premises let for any or all
    purposes specified in sub-section \ref{it:purpose} in the areas
    other than those specified in Schedule I and Schedule II.
  \end{clause} 

\end{subsectionlist}

\section{Exemption}
\begin{subsectionlist}
\item This Act shall not apply---
  \begin{clause}
  \item to any premises belonging to the Government or a local
    authority or apply as against the Government to any tenancy,
    licence or other like relationship created by a grant from or a
    licence given by the Government in respect of premises
    requisitioned or taken on lease or on licence by the Government,
    including any premises taken on behalf of the Government on the
    basis of tenancy or of licence or other like relationship by or in
    the name of any officer subordinate to the Government authorised
    in this behalf; but it shall apply in respect of premises let, or
    given on licence, to the Government or a local authority or taken
    on behalf of the Government on such basis by, or in the name of,
    such officer;
  \item to any premises let or sub-let to banks, or any Public Sector
    Undertakings or any Corporation established by or under any
    Central or State Act, or foreign missions, international agencies,
    multinational companies, and private limited companies and public
    limited companies having a paid up share capital of rupees one
    crore or more.

    \explanation{For the purpose of this clause the
      expression ``bank'' means,}---
    
    \begin{subclause}
    \item  the State Bank of India constituted under the State Bank of India Act,
      1955;\actref{23 of 1955} 
    \item  a subsidiary bank as defined in the State Bank of India (Subsidiary
      Banks) Act, 1959;\actref{38 of 1959}

    \item a corresponding new bank constituted under section 3 of the
      Banking Companies (Acquisition and Transfer of Undertakings)
      Act, 1970\actref{5 of 1970} or under section 3 of the Banking
      Companies (Acquisition and Transfer of Undertakings) Act,
      1980,\actref{4 of 1980} or
    \item any other bank, being a scheduled bank as defined in clause
      (e) of section 2 of the Reserve Bank of India Act,
      1934.\actref{2 of 1934}
    \end{subclause}
  \end{clause}

\item The State Government may direct that all or any of the
  provisions of this Act shall, subject to such conditions and terms
  as it may specify, not apply---
  \begin{clause}
  \item to premises used for public purpose of a charitable nature or to any
    class of premises used for such purpose;
  \item to premises held by a public trust for a religious or
    charitable purpose and let at a nominal or concessional rent;
  \item  to premises held by a public trust for a religious or
    charitable purpose and administered by a local authority; or
  \item to premises belonging to or vested in an university
    established by any law for the time being in force:
  \end{clause}
  provided that, before issuing any direction under this sub-section,
  the State Government shall ensure that the tenancy rights of the
  existing tenants are not adversely affected.

\item The expression ``premises belonging to the Government or a local
  authority'' in sub-section \ref{it:purpose} shall, notwithstanding
  anything contained in the said sub-section or in any judgement,
  decree or order of a court, not include a building erected on any
  land held by any person from the Government or a local authority
  under an agreement, lease, licence or other grant, although having
  regard to the provisions of such agreement, lease, licence or grant,
  the building so erected may belong or continue to belong to the
  Government or the local authority, as the case may be, and such
  person shall be entitled to create a tenancy in respect of such
  building or a part thereof.
\end{subsectionlist}

\section{Definitions}
\label{sec:def}

\begin{subsectionlist}
\item ``Government allottee'',--- \label{it:galot}
  \begin{clause}
    \item in relation to any premises requisitioned or continued under
    requisition which are allotted by the State Government for any
    non-residential purpose to any Department or office of the State
    Government or Central Government or any public sector undertaking
    or corporation owned or controlled fully or partly by the State
    Government or any Co-operative Society registered under the
    Maharashtra Co-operative Societies Act, 1960\actref{Mah. XXIV of
      1961} or any foreign consulate, by Land Requisition and Bombay
    Government Premises (Eviction) (Amendment) whatever name called
    and on the 7 th December 1996, being the date of coming into force
    of the Bombay Rents, Hotel and Lodging House Rates Control, Bombay
    Act, 1996, were in their occupation or possession, means the
    principal officer-in- charge of such office or department or
    public sector undertaking or corporation or society or consulate;
    and
    \label{it:galota}
  \item in relation to any premises requisitioned or continued under
    requisition which were allotted by the State Government for
    residential purpose to any person and on the 7 th December 1996,
    being the date of coming into force of the Bombay Rents, Hotel and
    Lodging House Rates Control, Bombay Land Requisition and Bombay
    Government Premises (Eviction) (Amendment) Act,
    1996,\actref{Mah. XVI of 1997} such person or his legal heir was
    in occupation or possession of such premises for his or such legal
    heir's own residence, means such person or legal heir ;
  \end{clause}
  
  
\item ``standard rent'' in relation to any premises
  means,---\label{it:stdrent}
  \begin{clause}
  \item where the standard rent is fixed by the Court or, as the case
    may be, the Controller under the Bombay Rent Restriction Act,
    1939, or the Bombay Rents, Hotel Rates and Lodging House Rates
    (Control) Act, 1944 or the Bombay Rents, Hotel and Lodging House
    Rates Control Act, 1947, or the Central Provinces and Berar
    Letting of Houses and Rent Control Order, 1949 issued under the
    Central Provinces and Berar Regulation of Letting of Accommodation
    Act, 1946 or the Hyderabad Houses (Rent, Eviction and Lease)
    Control Act, 1954, such rent plus an increase of 5 per cent. in
    the rent so fixed; or
  \item where the standard rent or fair rent is not so fixed, then
    subject to the provisions of section \ref{sec:courtfix},--- \label{it:stdrentunfix}
    \begin{subclause}
    \item the rent at which the premises were let on the first day of
      October 1987; or \label{it:oct87}
    \item where the premises were not let on the 1st day of October
      1987, or the rent at which they were last let before that day,
      plus an increase of 5 per cent.  in the rent of the premises let
      before the 1st day of October 1987, or \label{it:oct87no}
    \end{subclause}
  \item in any of the cases specified in section \ref{sec:courtfix},
    the rent fixed by the Court;
\end{clause}
\end{subsectionlist}


\chapter{Provisions Regarding Fixation of Standard Rent and Permitted Increase}
\section{Court may fix stanard rent and permitted increases in certain
  cases}
\label{sec:courtfix}
\begin{subsectionlist}
\item Subject to the provisions 9 in any of the following cases, the
  court may, upon an application made to it for the purpose or in any
  suit or proceedings, fix the standard rent at such amount as, having
  regard to the provisions of this Act and the circumstances of the
  case, the court, deems just,---
  \begin{clause}
  \item where the court is satisfied that there is no sufficient
    evidence to ascertain the rent at which the premises were let in
    any one of the cases mentioned in paragraphs \ref{it:oct87} and
    \ref{it:oct87no} of
    sub-clause \ref{it:stdrentunfix} of clause \ref{it:stdrent} of section
    \ref{sec:def} ; or
    \item  where by
    reasons of the premises having been let at one time as a whole or
    in parts and at another time, in parts or as a whole, or for any
    other reasons; or (c) where any premises have been or are let
    rent-free or at a nominal rent or for some consideration in
    addition to rent; or (d) where there is any dispute between the
    landlord and the tenant regarding the amount of standard rent.
  \end{clause}
\end{subsectionlist}

\section{State Government or Government allottee to become tenant of
  premises requisitioned or continued under requisition}

\begin{subsectionlist}
\item On the 7 th December 1996, that is the date of coming into force
  of the Bombay Rents, Hotel and Lodging House Rates Control, Bombay
  Land Requisition and Bombay Government Premises (Eviction)
  (Amendment) Act, 1996\actref{ Mah. XVI of 1997} (hereinafter in
  this section referred to as ``the said date''),--
  \begin{clause}
  \item the State Government, in respect of the premises requisitioned
    or continued under requisition and allotted to a Government
    allottee referred to in sub-clause \ref{it:galota} of clause
    \ref{it:galot} of section
    \ref{sec:def};
    and
  \item the Government allottee, in respect of the premises
    requisitioned or continued under requisition and allotted to him
    as referred to in sub-clause (b) of clause (2) of section
    \ref{sec:def}, shall, notwithstanding anything contained in this
    Act or in the \footnote{The short title of this Act was amended as
      the \emph{Maharashtra Land Requisition Act}, by Mah. 24 of 2012,
      Sch. entry 32, w.e.f. 1-5-1960.  } Bombay Land Requisition Act,
    1948, or in any other law for the time being in force or in any
    contract, or in any judgment,
  \end{clause}
\end{subsectionlist}

\end{document}



%%% Local Variables:
%%% mode: latex
%%% TeX-master: t
%%% End:
