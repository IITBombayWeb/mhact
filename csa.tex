%\documentclass[default]{mhact}
\documentclass[reprint]{mhact}
\begin{document}


    \actnamealias{1961: Mah. 24}{Maharashtra Act No 24 of 1961}

    \amendlastdate{As modified upto the 5th May 2012}
    \reprintYear{2012}
    \reprintPrice{75}

    \sorref{For Statement of Object and Reasons, see \emph{Maharashtra
      Government Gazette}, 1960, Part V, pages 270-273; for Report of
      the Select Committee see \emph{ibid.}, Part V, pages 432-517.}

    \assent{Received the assent of the President on the 4th of May
      1961; assent first published in the \emph{Maharashtra Government
      Gazzette}, Part IV, on the 9th day of May 1961.}

     \printRNInumber{MAHBIL/2009/40123}
      \printVarshSerial{२ }
      \printAnk{२१}
      \printAnkSerial{६}
      \printDiwas{सोमवार}
      \printTarikh{डिसेंबर ५, २०१६}
      \printMarathiTarikh{अग्रहायण १४, १९३७}
      \printPrice{३६}
      \printASKramank{४१}
      \printIntimation{}
    
      \shorttitle{Maharashtra Public Universities Act, 1960}
      \introducer{}
      \introdesig{}
      \psmla{}
    


      \longtitle{
        An Act to consolidate and amend the law relating to
        co-operative societies in the State of Maharashtra
        }
    

    %\workflowstep{As amended by the Joint Committee}

    \amendPrologue{%
        \begin{longtable}{llcll}
          Amended by Mah. & 5 of 1962 & & Amended by Mah. & 29 of
          1973\citenote{M13}{%
Maharashtra Ordinance No. VII of 1973 was repealed by
\actref{Mah.}{1969}{35}{4}} (30-5-1973) \\
          " " "  & 16 of 1969 & & " " " & 50 of 1977 (27-9-1977)  \\
          " " "  & 16 of 1969 & & " " " & 50 of 1977 (27-9-1977)  \\
        \end{longtable}

       % Even for a single column it is better to use a tabular for
       % getting the alignment right
       % Amended by Mah. 18 of 1984 (29-6-1984) \\
       % " " " Mah. 9 of 1985 (29-6-1985) \\
    }
    
    


    \startact 


    
    \enactingtext{ WHEREAS with a view to providing for the orderly
      developmet of the co-operative movement in the State of
      Maharashtra in accordance with the relevant directive principles
      of State policy enunciated in the Constitution of India, it is
      expedient to consolidate and amend the law relating to
      co-operative societies in that State ; It is hereby enacted in
      the Eleventh Year of the Republic of India as follows }
    

      \chapnumhead{1}{Preliminary}
      

      \secnumhead{1}{Short title and Commencement}
      

      \begin{subsectionlist}

        \lblitem{1} This Act may be called the Maharashtra
        Co-operative Societies Act, 1960.

        \lblitem{2} It extends to the whole of the State of
        Maharashtra.

        \lblitem{3} It shall come into force on such date as the State
        Government may, by notification in the Official Gazette,
        appoint.

       \end{subsectionlist}
    
      \secnumhead{2}{Definitions}
      
      In this Act, unless the context otherwise requires,---


      \begin{subsectionlist}
        \lblitem{1} ``agricultural marketing society'' means a
        society---

        \begin{clause}

        \lblitem{a} the object of which is the marketing of
        agricultural produce and the supply of implements and other
        requisites for agricultural production, and

        \lblitem{b} not less than
        three-fourths of the members of which are agriculturists, or
        societies formed by agriculturists ;
      \end{clause}
      
      \lblitem{2}  ``apex society'' means a society,---
      \begin{clause}
        \lblitem{a} the area of operation of which extends
        to the whole of the State of Maharashtra,

        \lblitem{b} the main object of which is to promote the
        principal objects of the societies affiliated to it as members
        and to provide for the facilities and services to them, and

        \lblitem{c} which has been classified as an
        apex society by the Registrar ;].
      \end{clause}
      

      \lblitem{3} *  * * * * *

      \lblitem{4} ``bonus'' means payment made in cash or kind out of
      the profits of a society to a member, or to a person who is not
      a member, on the basis of his contribution (including any
      contribution in the form of labour or service) to the business
      of the society, and in the case of a farming society, on the
      basis both of such contribution and also the value or income or,
      as the case may be, the area of the lands of the members brought
      together for joint cultivation as may be decided by the society
      [but does not include any sum paid or payable as bonus to any
      employee of the society under the Payment of Bonus Act,1965 ;].
    \end{subsectionlist} 

    \chapnumhead{2}{Registration}
    \secnumhead{3}{Registrar [and his subordinates]}
     \secnumhead{3A}{Temporary vacancies}

     \secnumhead{4}{Societies which may be registered}

     \secnumhead{7}{ Power to exempt societies 3 [or class of societies]
       from conditions as to registration. Application for
       registration. }

    \secnumhead{11}{ Power of Registrar to decide certain questions }

    When, 3 * * * * * * any question arises whether a person is an
    agriculturist or not, or whether any person resides in the area of
    operation of the society or not, 4 [or whether a person is or is
    not engaged in or carrying on any profession, business or
    employment, or whether a person belongs or does not belong to such
    class of persons as declared under sub-section (1-A) of section 22
    and has or has not incurred a disqualification under that
    sub-section,] such question shall be decided by the Registrar 5
    [and his decision shall be final, but no decision adverse to any
    such person shall be given without giving him an opportunity of
    being heard.]


    \chapnumhead{14}{Miscellaneous}

    \secnumhead{155}{Recovery of sums due to Government}

    \secnumhead{156}{Registrar's powers to recover certain sums by
      attachment and sale of property.}

    \secnumhead{167}{Companies Act not to apply} For the removal of
    doubt, it is hereby declared that the provisions of the Companies
    %Act, 1956\citenote{M1671}{\actref{}{1956}{1}} shall not apply to
    societies registered, or deemed to be registered, under this Act.

    
\end{document}

