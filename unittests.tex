%\documentclass[default]{mhact}
\documentclass[default]{report}
\usepackage{times}
\usepackage{titlesec}
\usepackage{manyfoot}
\usepackage{lipsum}
\usepackage{longtable}


\usepackage{soul}

\RequirePackage{xcolor}
%\newcommand{\href}[2]{#2}



\newcommand{\ins}[1]{\ul{#1}}
\newcommand{\del}[1]{\st{#1}}

% \newcommand{\ins}[1]{\protect\ul{#1}}
% \newcommand{\del}[1]{\protect\st{#1}}

% \newcommand{\ins}[1]{\emph{#1}}
% \newcommand{\del}[1]{[#1]}

% \newcommand{\ins}[1]{\underline{#1}}
% \newcommand{\del}[1]{[#1]}
\newcommand{\subs}[2]{\del{#1} \ins{#2}}

\newcommand{\chapnumhead}[2]{%

  
  \titleformat{\chapter}[display]
  {\centering\large\bfseries\scshape}
  {\chaptertitlename \space#1}
  {\baselineskip}{}{}
  \chapter{\texorpdfstring{#2}{}}
  %% for soul to work correctly in headings with hyperref
  %% https://tex.stackexchange.com/a/240438/33945

  \titleformat{\chapter}[display]
  {\centering\large\bfseries\scshape}
  {\chaptertitlename \space\thechapter}
  {\baselineskip}{}{}
}

\newcommand{\secnumhead}[2]{%
	\titlelabel{#1\quad}
	\section{\texorpdfstring{#2}{}}
	\titlelabel{\thechapter\quad}
}



\usepackage{enumitem}
\newcommand{\lblitem}[1]{%
  \def\lblnum{#1} \item
}

\newenvironment{subsectionlist}[1][99]{%
  %\def{\lblnum}{1}
  \begin{enumerate}[label=(\emph{\protect\lblnum)}]
    \settowidth{\leftmargin}{#1}
  }{\end{enumerate}}


\usepackage[hang]{footmisc}



\RequirePackage[bookmarks,%
            breaklinks,%
            backref=false,%
            pdfhighlight=/I,%
            pdffitwindow=true,%
            pdfstartview=Fit,%
            pdfcenterwindow=true,%
            colorlinks,
            linkcolor={red!50!black},
            urlcolor={blue!80!black},
            pdfpagelabels,
	        pdfusetitle]
            {hyperref}



%% Footnotes (manyfoot)

%\SetFootnoteHook{\hangindent=2em\noindent}
% \newcommand{\Ffootnoterule}{\noindent\rule{\linewidth}{0.5pt}\par \noindent\footnotesize
%   Amending Notes: \vspace{2mm}}
% \SelectFootnoteRule[1]{F}
%\SelectFootnoteRule[1]{F}[\noindent\footnotesize \smash{Amending Notes:}\vspace{2mm}]


%\DeclareNewFootnote{F}
%\SetFootnoteHook{\hangindent=2em\noindent}
% \SelectFootnoteRule[1]{M}[\noindent\footnotesize \smash{Act Refs.:}\vspace{2mm}]
% \DeclareNewFootnote{M}[alph]

\newcommand{\Ffootnoterule}{\noindent\rule{0.5\linewidth}{0.5pt}\par\noindent\footnotesize Amendments (Textual): \vspace{2mm}}
\SelectFootnoteRule[1]{F}
\DeclareNewFootnote{F}

\newcommand{\Cfootnoterule}{\noindent\rule{0.5\linewidth}{0.5pt}\par\noindent\footnotesize Modifications etc. (not altering text): \vspace{2mm}}
\SelectFootnoteRule[1]{C}
\DeclareNewFootnote{C}


\newcommand{\Ifootnoterule}{\noindent\rule{0.5\linewidth}{0.5pt}\par\noindent\footnotesize Commencement Information: \vspace{2mm}}
\SelectFootnoteRule[1]{I}
\DeclareNewFootnote{I}

\newcommand{\Mfootnoterule}{\noindent\rule{0.5\linewidth}{0.5pt}\par\par\noindent\footnotesize
  Marginal Citations: \vspace{2mm}}
%\SelectFootnoteRule[1]{R}[\noindent\footnotesize \smash{Act Refs.:}\vspace{2mm}]
\SelectFootnoteRule[1]{M}
\DeclareNewFootnote{M}[alph]


\usepackage{etoolbox}
\makeatletter
\newcommand\manyfoottarget{\makebox[0pt][r]{\hypertarget{Hfootnote.\theHfootnote
}\quad}}
\patchcmd\MFL@fnoteplain{\rule}{\manyfoottarget\rule}{}{\fail}
\makeatother
%% Needed to get hyperlinks working with manyfoot
%% https://tex.stackexchange.com/a/498999/33945


\newcommand{\actref}[4]{%
  %\href{http://legislation.maharashtra.gov.in/acts/#1/#2/#3}{[{#1}{#3} of {#2}]}}
   \ifx\hfuzz#1\hfuzz % https://tex.stackexchange.com/a/53091/33945
     \ifx\hfuzz#4\hfuzz % https://tex.stackexchange.com/a/53091/33945
       \href{http://legislation.maharashtra.gov.in/acts/india/#2/#3}{{#3} of {#2}}
     \else
       \href{http://legislation.maharashtra.gov.in/acts/india/#2/#3/#4}{{#3}
         of {#2}, {#4}}
     \fi
   \else
     \ifx\hfuzz#4\hfuzz % https://tex.stackexchange.com/a/53091/33945
       \href{http://legislation.maharashtra.gov.in/acts/#1/#2/#3}{{#1} {#3} of {#2}}
     \else
      \href{http://legislation.maharashtra.gov.in/acts/#1/#2/#3/#4}{%
        {#1} {#3} of {#2}, {#4}} 
     \fi
   \fi
}

\renewcommand{\thefootnoteM}{M\arabic{footnoteM}}
\renewcommand{\thefootnoteC}{C\arabic{footnoteC}}
\renewcommand{\thefootnoteI}{I\arabic{footnoteI}}
\renewcommand{\thefootnoteF}{F\arabic{footnoteF}}

\newcommand{\citenote}[2]{%
  %% Takes two arguments: \citenote{uniqkey}{actref}
  %% inserts the marker at the location as well as the foonote
  \footnoteM{#2\label{fnm:#1}}}

\newcommand{\Cnote}[2]{%
  %% Takes two arguments: \Cnote{uniqkey}{Foonote Text}
  %% Does not insert the marker. Has to be explicitly inserted using a
  %% \Cnoteref{} command. This is so that the same marker can be
  %% referred in multiple locations
  %% \Cnote{} and \Cnoteref{} must appear in the same page
  \stepcounter{footnoteC}
  \FootnotetextC{\thefootnoteC}{#2\label{fnc:#1}}}

  %\footnotetextC[\thefootnoteC]{#2\label{fnc:#1}}}
  %% plain footnotetextC throws an error in the presence of hyperref
  %% when thefootnote has a prefix C

\newcommand{\Inote}[2]{%
  %% Takes two arguments: \Inote{uniqkey}{Foonote Text}
  %% Does not insert the marker. Has to be explicitly inserted using a
  %% \Inoteref{} command. This is so that the same marker can be
  %% referred in multiple locations
  %% \Inote{} and \Inoteref{} must appear in the same page
  \stepcounter{footnoteI}
  \FootnotetextI{\thefootnoteI}{#2\label{fni:#1}}}

\newcommand{\Fnote}[2]{%
  %% Takes two arguments: \Fnote{uniqkey}{Amendment Note}
  %% Does not insert the marker. Has to be explicitly inserted using a
  %% pair of \FnoteBegin{} \FnoteEnd commands.
  %% This is so that the same marker can be
  %% referred in multiple locations
  %% \Fnote{} and \FnoteBegin{} must appear in the same page
  %% \FnoteEnd{} can appear in another page
  \stepcounter{footnoteF}
  \FootnotetextF{\thefootnoteF}{#2\label{fnf:#1}}}





\newcommand{\Cnoteref}[1]{\protect\footref{fnc:#1}}
\newcommand{\Inoteref}[1]{\footref{fni:#1}}
\newcommand{\FnoterefBegin}[1]{\footref{fnf:#1}[\ignorespaces}
\newcommand{\FnoterefEnd}[1]{]\footref{fnf:#1}}


%\newcommand{\citenote}[2]{#1:#2}
%\newenvironment{amendins}[1]{\footnoteF{#1}[}{]$_{\mathrm{\thefootnoteF}}$}
\newenvironment{amendins}[2]{\def\mylab{cf:#1}\footnoteF{#2\label{\mylab}}[\ignorespaces}{\ifhmode\unskip\fi]\footref{\mylab}}

%\newenvironment{amendins}[2]{\footnoteF{#2}[\ignorespaces}{ \ifhmode\unskip\fi]\Footnotemark\thefootnoteF}

%\newenvironment{amendins}[2]{[}{]}

%\newcommand\amenddel[1]{\footnoteF{#1}[ * * * ]\Footnotemark\thefootnoteF}

\newcommand{\amenddel}[2]{\def\mylab{cf:#1}\footnoteF{#2\label{\mylab}}[ * * * * ]\footref{\mylab}}

%\newcommand\amenddel[1]{}{}

% \usepackage{verbatim}
%   %*****\begin{comment}}{\end{comment}\Footnotemark\thefootnoteF]}
% does not work: file ended while scanning \next 

% \newenvironment{amenddel}[1]{[\footnoteF{#1} 
%   *****}{\Footnotemark\thefootnoteF]}



\begin{document}

\tableofcontents



\chapnumhead{\ins{2}}{\ins{Preliminary}}
\Inoteref{I1}
\Inote{I1}{For statement of
  object and reasons see Mah. Gazzette}



\section{Introduction}


1939\Cnoteref{C1},
\Cnote{C1}{Extended by \actref{Mah.}{1983}{30}{}}
or the Bombay Rents\Cnoteref{C1}, Hotel Rates and Lodging House Rates
(Control) Act, 1944\citenote{M10}{\actref{Bom.}{1944}{22}{}} or the Bombay Rents, Hotel and Lodging House Rates
Control Act, 1947\citenote{M11}{\actref{Mah.}{1947}{32}{3(c)}}, or the
Central Provinces and Berar Letting of Houses and Rent Control Order,
1949 issued under the Central Provinces and Berar Regulation of
% \begin{amendins}{F12}{These words were inserted by \actref{Mah.}{1993}{27}{2(h)}}
%   Letting of Accommodation Act, 1946 or the Hyderabad Houses (Rent,
%   Eviction and Lease) Control Act, 1954,
% \end{amendins}
\Fnote{F12}{These words were inserted by \actref{Mah.}{1993}{27}{2(h)}}
\FnoterefBegin{F12}Letting of Accommodation Act, 1946 or the Hyderabad Houses (Rent,
Eviction and Lease) Control Act, 1954,\FnoterefEnd{F12}

  such rent plus an increase of 5
per cent. in \newcounter{ssec}
\begin{list}{\arabic{ssec}.}{%
    \usecounter{ssec}
    \settowidth\labelwidth{ZZAA.}
    \leftmargin\labelwidth
    % There must be a better way to increment it three times
    \advance\leftmargin\labelsep
    \advance\leftmargin\labelsep 
    \advance\leftmargin\labelsep 
  }
\item This Act shall, in the first instance,
  \Fnote{F22}{These words were substituted for the words ``when a
      society has passed a resolution to change'' by
      \actref{Mah.}{1986}{20}{99ab}.
    }
  \FnoterefBegin{F22}apply to premises let for
  the purposes of residence,\FnoterefEnd{F22}
  education, business, trade or storage in
  the areas specified in Schedule I and Schedule II.
\item [\ins{1ZAA.}] This is a sample insert subsection
\item [\del{1ZAA.}] \del{This has been deleted}


\Fnote{F41}{Clause 2 was deleted by \actref{Mah.}{1993}{23}{5c}}
\FnoterefBegin{F41} 
* * *
\FnoterefEnd{F41} 

\item \Fnote{F43}{The word ``Nevertheless'' was deleted by
    \actref{Mah.}{2012}{33}{29f}}
  \FnoterefBegin{F43} 
  * * *
  \FnoterefEnd{F43} 
  Notwithstanding anything contained in sub-section~1 it shall
  also apply to the premises or as the case may be, houses let out in
  the areas to which the Bombay Rents, Hotel and Lodging House Rates
  Provinces and \del{Bayr} \ins{Berar} \subs{Let}{Letting} of Houses
  and Rent Control Order, 1949

    
\end{list}


	

\secnumhead{\del{2} \ins{3}}{\del{Foreground} \ins{Background}}

\Cnote{C11}{Extended by \actref{Mah.}{2011}{84}{}}
\Cnote{C2}{Extended by \actref{Mah.}{1991}{93}{}}

(Control) Act, 1944\citenote{M23}{\actref{Mah.}{1944}{20}{}} or the
Bombay Rents, Hotel and Lodging House Rates Control Act, 1947,
\Fnote{F32}{Inserted by \actref{Mah.}{2014}{20}{3c}}
\FnoterefBegin{F32}or the
Central Provinces and Berar Letting of Houses and Rent Control Order,
1949 issued\Cnoteref{C11}
\Fnote{F33}{Substituted the words ``and other similar acts'' by
    \actref{Mah.}{2018}{10}{8c}}
\FnoterefBegin{F33} under the Central Provinces and Berar Regulation of
  Letting of Accommodation Act, 1946
\FnoterefEnd{F33}
  or the Hyderabad Houses (Rent,
Eviction and Lease) Control Act, 1954,\Cnoteref{C2}
\FnoterefEnd{F32}
such rent plus an increase of 5 
per cent. in
\begin{enumerate}
\item This Act shall, in the first instance, apply to premises let for
  the purposes of residence, education, business, trade or storage in
  the areas specified in Schedule I and Schedule II.
  \label{it:purpose}
\item[1ZAA.] This is a sample insert subsection

\item[] This is a sample insert subsection without label
\item Notwithstanding anything contained in sub-section
  \ref{it:purpose}, it shall
  also apply to the premises or as the case may be, houses let out in
  the areas to which the Bombay Rents, Hotel and Lodging House Rates
  Provinces and Berar Letting of Houses and Rent Control Order, 1949

\FnoterefBegin{F45}
  \Fnote{F45}{%
    The following section was deleted by \actref{Mah.}{1972}{20}{12m}
    \begin{enumerate}
    \item \lipsum[1]
    \item \lipsum[2-5]
    \end{enumerate}
  }
  * * * \FnoterefEnd{F45} 


\end{enumerate}

\clearpage
\section{Prologue}

\begin{longtable}{llcll}
  Amended by Mah. & 5 of 1962\footnote{Default} & & Amended by Mah. & 29 of
  1973\citenote{M13}{%
    Maharashtra Ordinance No. VII of 1973 was repealed by
    \actref{Mah.}{1969}{35}{4}}
  (30-5-1973) \\
  " " "  & 16 of 1969 & & " " " & 50 of 1977 (27-9-1977) 
\end{longtable}
      

\section{Literature}
\begin{subsectionlist}
\lblitem{1} Within a period of sixty days from the date of commencement of 
this Act or the date on which establishment commences its business, the 
employer of every establishment employing ten or more workers shall submit
\citenote{M43}{\actref{Mah.}{1993}{23}{4k}} application online in a prescribed form for registration to the Facilitator of
1973\citenote{M45}{%
Maharashtra Ordinance No. VII of 1973 was repealed by
\actref{Mah.}{1969}{35}{4}}
the local area concerned, together with such fees and such self-declaration
and self-certified documents as may be prescribed, containing---
\lblitem{\ins{2}} the name of the employer and the manager, if any;
\lblitem{\subs{2}{3}} the postal address of the establishment;
\lblitem{\subs{3}{4}} the name, if any, of the establishment;
\lblitem{\del{4}} the name, if any, of the establishment;
\end{subsectionlist}

\end{document}

%%% Local Variables:
%%% mode: latex
%%% TeX-master: t
%%% End:
