%\documentclass[default]{mhact}
\documentclass[default]{report}
\usepackage{times}
\usepackage{titlesec}

\newcommand{\ins}[1]{\underline{#1}}
\newcommand{\del}[1]{[#1]}
\newcommand{\subs}[2]{\del{#1} \ins{#2}}


\newcommand{\sectionedit}[2]{%
	\titlelabel{#1\quad}
	\section{#2}
	\titlelabel{\thesection\quad}
}

\newcommand{\chapteredit}[2]{%
	\titlelabel{#1\quad}
	\chapter{#2}
	\titlelabel{\thechapter\quad}
}

\usepackage{enumitem}
\newcommand{\lblitem}[1]{%
  \def\lblnum{#1} \item
}

\newenvironment{subsectionlist}[1][99]{%
  %\def{\lblnum}{1}
  \begin{enumerate}[label=(\emph{\protect\lblnum)}]
    \settowidth{\leftmargin}{#1}
  }{\end{enumerate}}


\begin{document}


\chapteredit{\ins{2}}{\ins{Preliminary}}

\section{Introduction}

1939, or the Bombay Rents, Hotel Rates and Lodging House Rates
(Control) Act, 1944 or the Bombay Rents, Hotel and Lodging House Rates
Control Act, 1947, or the Central Provinces and Berar Letting of
Houses and Rent Control Order, 1949 issued under the Central Provinces
and Berar Regulation of Letting of Accommodation Act, 1946 or the
Hyderabad Houses (Rent, Eviction and Lease) Control Act, 1954, such
rent plus an increase of 5 per cent. in
\newcounter{ssec}
\begin{list}{\arabic{ssec}.}{%
    \usecounter{ssec}
    \settowidth\labelwidth{ZZAA.}
    \leftmargin\labelwidth
    % There must be a better way to increment it three times
    \advance\leftmargin\labelsep
    \advance\leftmargin\labelsep 
    \advance\leftmargin\labelsep 
  }
\item This Act shall, in the first instance, apply to premises let for
  the purposes of residence, education, business, trade or storage in
  the areas specified in Schedule I and Schedule II.
  \label{it:purpose}
\item [\ins{1ZAA.}] This is a sample insert subsection
\item [\del{1ZAA.}] \del{This has been deleted}
\item Notwithstanding anything contained in sub-section
  \ref{it:purpose}, it shall
  also apply to the premises or as the case may be, houses let out in
  the areas to which the Bombay Rents, Hotel and Lodging House Rates
  Provinces and \del{Bayr} \ins{Berar} \subs{Let}{Letting} of Houses and Rent Control Order, 1949
\end{list}


	

\sectionedit{\del{2} \ins{3}}{\del{Bg} \ins{Background}}


(Control) Act, 1944 or the Bombay Rents, Hotel and Lodging House Rates
Control Act, 1947, or the Central Provinces and Berar Letting of
Houses and Rent Control Order, 1949 issued under the Central Provinces
and Berar Regulation of Letting of Accommodation Act, 1946 or the
Hyderabad Houses (Rent, Eviction and Lease) Control Act, 1954, such
rent plus an increase of 5 per cent. in
\begin{enumerate}
\item This Act shall, in the first instance, apply to premises let for
  the purposes of residence, education, business, trade or storage in
  the areas specified in Schedule I and Schedule II.
  \label{it:purpose}
\item[1ZAA.] This is a sample insert subsection

\item[] This is a sample insert subsection without label
\item Notwithstanding anything contained in sub-section
  \ref{it:purpose}, it shall
  also apply to the premises or as the case may be, houses let out in
  the areas to which the Bombay Rents, Hotel and Lodging House Rates
  Provinces and Berar Letting of Houses and Rent Control Order, 1949

\end{enumerate}

\section{Literature}
\begin{subsectionlist}
\lblitem{1} Within a period of sixty days from the date of commencement of 
this Act or the date on which establishment commences its business, the 
employer of every establishment employing ten or more workers shall submit
application online in a prescribed form for registration to the Facilitator of
the local area concerned, together with such fees and such self-declaration
and self-certified documents as may be prescribed, containing---
\lblitem{\ins{2}} the name of the employer and the manager, if any;
\lblitem{\subs{2}{3}} the postal address of the establishment;
\lblitem{\subs{3}{4}} the name, if any, of the establishment;
\lblitem{\del{4}} the name, if any, of the establishment;
\end{subsectionlist}

\end{document}

%%% Local Variables:
%%% mode: latex
%%% TeX-master: t
%%% End:
