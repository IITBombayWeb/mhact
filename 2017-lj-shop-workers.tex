%%% Sample New Act covering all the necessary features

\documentclass[billwithdocket]{mhact}
%\documentclass[gaz5a]{mhact}
%\documentclass[gaz8]{mhact}
% \documentclass[reprint]{mhact}



\begin{document}

%% Printer Metadata %%
\printRNInumber{MAHBIL/2009/40123}
\printVarshSerial{३}
\printAnk{२०}
\printAnkSerial{२}
\printDiwas{मंगळवार}
\printTarikh{ऑगस्ट ८, २०१७}
\printMarathiTarikh{श्रावण १७, शके १९३९}
\printPrice{३६.00}

\printASKramank{४८}


\printIntAuth{Maharashtra Legislature Secretariat}
\printIntimation{The following Bill was introduced in the Maharashtra
  Legislative Assembly on the 8th August 2017 is published under Rule
  117 of the Maharashtra Legislative Assembly Rules
}


\actnamealias{2017: Mah. 54}{L.A. Bill 54 of 2017}
\shorttitle[Shops and Establishments Regulation ... Bill]{
Maharashtra Shops and Establishments
(Regulation of Employment and Conditions of Service) Bill, 2017
}

\introducer{Shri Sambhaji Patil Nilangekar}
\introdesig{Minister for Labour}
\introdate{3rd August 2017}

%\assemblydate{9th August 2017}
%\councildate{11th August 2017}

\workflowstep{As passed by the Legislative Assembly with Amendment on the
19th December 2017.}
\workflowstep{As passed by the Legislative Council with Amendment on the
22nd December 2017.}
\workflowstep{As passed by the Legislative Assembly on 8th March 2018 by
giving concurrance to the amendment made by the
Legislative Council alongwith additional amendments.}
\workflowstep{As passed by the Legislative Assembly for the second time on the
13th July 2018.}
\workflowstep{As passed by the Legislative Councl on the 20th July 2018.}


\psmla{Dr Anant Kalse}



%\actedition{As modified up to 21 May 2013} \actedyear{2013}
%\price{Rs 23}

%\maketitle


\startact



% \assent{%
%   This Act received the assent of the President on 8 March 2000; assent
%   was first published in the Maharashtra Government Gazette, Part-IV, on 
%   10 March 2000.}

\longtitle{
  A Bill to provide for the regulation of conditions of
  employment and other conditions of service of workers employed in
  shops, residential hotels, restaurants, eating houses, theatres,
  other places of public amusement or entertainment and other
  establishments and for matters connected therewith or incidental
  thereto.
}



\enactingtext{%
  WHEREAS it is expedient to provide for the regulation of employment
  and other conditions of service of workers employed in shops,
  residential hotels, restaurants, eating houses, theatres, other
  places of public amusement or entertainment and other establishments
  and for matters connected therewith or incidental thereto; it is
  hereby enacted in the Sixty-eighth Year of the Republic of India as
  follows}

\chapter{Preliminary}

\section{Short title, extent, application and commencement}

\begin{subsectionlist}
\item This Act may be called the Maharashtra Shops and Establishments
(Regulation of Employment and Conditions of Service) Act, 2017.
\item It extends to the whole of the State of Maharashtra.
\item The provisions of this Act, except section 7, shall apply to the
establishments employing ten or more workers and the provisions of section
7 shall apply to the establishments employing less than ten workers.
\item It shall come into force on such date as the State Government may by
notification in the Official Gazette appoint.
\end{subsectionlist}


\section{Definitions}
\intro{
In this Act, unless the context otherwise requires,
}
\begin{subsectionlist}
\item ``Chief Facilitator'' means the Chief Facilitator appointed as such
under section 28 of this Act;
\item ``day'' means the period of twenty-four hours beginning at
midnight;
\item ``employer'' means a person owning or having ultimate control
over the affairs of an establishment, and includes,--
\begin{clause}
\item in the case of a firm or association of individuals, a partner
or members of the firm or association;
\item in the case of a company, a director of the company;
\item in the case of an establishment owned or controlled by the
Central Government or a State Government or any local authority,
the person or persons appointed to manage the affairs of such
establishment by the Central Government or the State Government
or the local authority, as the case may be;
\end{clause}
\item ``establishment'' means an establishment which carries on, any
business, trade, manufacture or any journalistic or printing work, or
business of banking, insurance, stocks and shares, brokerage or produce
exchange or profession or any work in connection with, or incidental or
ancillary to, any business, trade or profession or manufacture; and
includes establishment of any medical practitioner (including hospital,
dispensary, clinic, polyclinic, maternity home and such others), architect,
engineer, accountant, tax consultant or any other technical or
professional consultant; and also includes a society registered under
the Societies Registration Act, 1860,
 \actref{}{1860}{41}{}
  and a charitable or other trust,
whether registered or not, which carries on, whether for purposes of
gain or not, any business, trade or profession or work in connection with
or incidental or ancillary thereto; and includes shop, residential hotel,
restaurant, eating house, theatre or other place of public amusement or
entertainment; to whom the provisions of the Factories Act, 1948 does
not apply ; and includes such other establishment as the State
Government may, by notification in the Official Gazette, declare to be an
establishment for the purposes of this Act;
\item ``Facilitator'' means a Facilitator appointed under section 28 of
this Act;
\item ``Factory'' means any premises which is a factory within the
meaning of clause (m) of section 2 of the Factories Act, 1948 or which is
deemed to be a factory under section 85 of the said Act;

\item ``holiday'' means a day on which a worker shall be given a weekly
off under the provisions of this Act;
\item ``leave'' means a leave provided for in Chapter IV of this Act;
\item ``local area'' means any area or combination of areas to which
this Act applies;
\item ``local authority'' means the Municipal Corporation of Brihan
Mumbai constituted or deemed to have been constituted under the
Mumbai Municipal Corporation Act, Corporations constituted or deemed
to have been constituted under the Maharashtra Municipal Corporations
Act and the Municipal Councils constituted or deemed to have been
constituted under the Maharashtra Municipal Councils, Nagar
Panchayats and Industrial Townships Act, 1965, and includes any other
body which the State Government may, by notification in the Official
Gazette, declares to be a local authority for the purposes of this Act;
\item ``Manager'' means a person mentioned in the application under
section 6 of this Act ;
\item ``member of the family of an employer'' means the wife, husband,
son, daughter, father, mother, brother or sister of an employer who lives
with and is dependent on such employer;
\item ``opened'' means opened for the service of any customer, or for
any business of the establishment, or for work, by or with the help of any
worker of or connected with the establishment;
\item ``period of work'' means the time during which a worker is at
the disposal of the employer;
\item ``prescribed'' means prescribed by rules made under this Act;
\item ``prescribed authority'' means the Commissioner of Labour for
the purposes of this Act;
\item ``register of establishment'' means a register maintained for the
registration of establishments under this Act, either manually or in
electronic format;
\item ``registration certificate'' means a certificate of the registration
of an establishment;
\item ``residential hotel'' means any premises used for the reception
of guests and travellers desirous of dwelling or sleeping therein and
includes residential club;
\item ``restaurant or eating house'' means any premises, in which,
wholly or principally the business of the supply of meal or refreshments
to the public or a class of the public for consumption on the premises is
carried on;
\item ``shop'' means any premises where goods are sold, either by
retail or wholesale or where services are rendered to customers, and
includes an office, a store-room, godown, warehouse or work place,
whether in the same premises or otherwise, mainly used in connection
with such trade or business, but does not include a factory;
\item ``spread over'' means the period between the commencement
and the termination of the work of a worker on any day;
\item ``theatre'' includes any premises intended principally or wholly
for the exhibition of pictures or other optical effects by means of a
cinematograph or other suitable apparatus or for dramatic performances
or for any other public amusement or entertainment;
\item ``wages'' means wages as defined in the Payment of Wages Act,
1936;
\item ``week'' means the period of seven days beginning at midnight
of Saturday;
\item "worker" means any person (except an apprentice under the
Apprentices Act, 1961) employed to do any manual, unskilled, skilled
technical, operational or clerical work for hire or reward, whether the
terms of employment be express or implied.

\end{subsectionlist}


\section{Act not to apply to certain establishments and persons}

\intro{The provisions of this Act shall not apply to,}
\begin{subsectionlist}
\item Establishments of the Central and State Government;
\item Establishments of Local Authorities;
\item Establishment of Mumbai Port Trust;
\item Establishment of Railway Administration;
\item Offices of Reserve Bank of India;
\item Offices of the Trade Commissioner and of Consular officers and
other Diplomatic representatives of Foreign Government;
\item Offices of Air Service Companies;
\item Establishments used for treatment or care of infirm, destitute
or mentally unfit;
\item Establishments pertaining to any kind of educational activities
(excepting those where coaching or tuition classes are conducted by
individual persons or any institutions other than those,-
\begin{clause}
\item affiliated to any university established by law, or
\item recognised by the Divisional Boards under the Maharashtra
Secondary and Higher Secondary Education Boards Act, 1965, or
\item recognised by the Directorate of Education or the
Directorate of Technical Education as a private secondary or
technical high school, Industrial Training Institute (I.T.I.),
Polytechnic, Engineering Colleges or other technical institutions
conducting courses recognised by Government) ;
\end{clause}
\item High Court Law Libraries and other Courts Law Libraries;
\item A worker occupying position of confidential, managerial or
supervisory character in an establishment, a list of which shall be
displayed on the website of establishments and in absence of the website
at a conspicuous place in the establishment and a copy thereof shall be
sent to the Facilitator;
\item A worker whose work is inherently intermittent;
\item A member of the family of an employer.
\end{subsectionlist}


\section{Application of Act to other establishments and workers }
\begin{subsectionlist}
\item Notwithstanding anything contained in this Act, the State
Government may, by notification in the \emph{Official Gazette}, declare any
establishment or class of establishments to which, or any worker or person
or class of workers or persons to whom, this Act or any of the provisions
thereof does not for the time being apply, to be an establishment or class of
establishments or a worker or a person or class of workers or persons to
which or whom this Act or any provisions thereof with such modifications or
adaptations as may in the opinion of the State Government be necessary
shall apply from such date as may be specified in the notification.
\item On such declaration under sub-section (1), any such establishment
or class of establishments or such worker or person or class of workers or

\end{subsectionlist}

\section{Suspension of all or any of provisions of this Act}
The State Government may, by notification in the \emph{Official
  Gazette}, suspend the operation of all or any of the provisions of
this Act for such period and subject to such conditions as it deems
fit on account of any holidays or occasions

%\includepdf[pages=-,pagecommand={\pagestyle{fancy}},templatesize={5cm}{10cm},frame]{act-ma.pdf}

%\includepdf[landscape,pages=-,pagecommand={\pagestyle{fancy}},templatesize={20cm}{20cm}]{sample-financebill.pdf}

\chapter{Registration of Establishments}

\section{Registration of establishments}

\begin{subsectionlist}
\item Within a period of sixty days from the date of commencement of 
this Act or the date on which establishment commences its business, the 
employer of every establishment employing ten or more workers shall submit
application online in a prescribed form for registration to the Facilitator of
the local area concerned, together with such fees and such self-declaration
and self-certified documents as may be prescribed, containing---
\begin{enumerate}[label=(\alph*)]
\item the name of the employer and the manager, if any;
\item the postal address of the establishment;
\item the name, if any, of the establishment;
\item the actual nature of the business of the establishment; and
\item such other particulars as may be prescribed:
\end{enumerate}
Provided that, nothing contained hereinabove shall apply to the
establishments already having valid registration or renewal under the
Maharashtra Shops and Establishments Act    \actref{Mah.}{1948}{79}{}
until expiry of their registration or renewal.


\item On receipt of the application along with documents and the fees
  online, the Facilitator shall, register the establishment in the
  register of establishments in such manner as may be prescribed and
  shall issue online, in a prescribed form, a registration certificate
  along with the Labour Identification Number (LIN) to the employer
  within the prescribed time limit.  The Facilitator shall verify the
  correctness of the application and documents attached thereto within
  such time as may be prescribed. The registration certificate shall
  be produced whenever it is demanded by the Facilitator.

\end{subsectionlist}


\attachpdf[landscape]{Finance Bill}{sample-financebill.pdf}

%% Statement of Objects and Reasons
\begin{sor}
\item The Maharashtra Shops and Establishments Act (LXXIX of 1948) is
  enacted to consolidate and amend the law relating to the regulation
  of conditions of work and employment in shops, commercial
  establishments, residential hotels, restaurants, eating houses,
  theatres, other places of public amusement or entertainment and
  other establishments.

\item The recent information and technology have revolutionized the
  mode of trading whereby it is possible to sell goods and services
  online without any physical, and geographical limitations and time
  limitations being available for twenty-four hours. Therefore, the
  provisions of the said Act of keeping a shop or establishment closed
  for a business on one day of the week and to restrict the opening
  and closing hours of establishments have become obsolete.  It has
  become necessary to provide even platform for offline business to
  compete with online business and to permit to operate shops and
  establishments for twenty-four hours and all days in a week. It will
  help employment generation at large and to increase Gross Domestic
  Product.

\item In line with the "ease of doing business" policy of the
  Government, it is necessary that the marginal and small
  establishments engaging less than ten employees need to be
  liberalized from registration under the Act and to provide all the
  services under the Act online based on self-declaration and
  self-certified documents.  It is also necessary that the employees
  in the shops and establishments should have uniform working
  conditions. All the welfare measures should be applicable to workers
  so as to improve their health and well being which in turn will
  increase their productivity. Also due to the increase in the
  literacy percentage of women, numerous avenues for job and
  employment for women workers are available. Allowing them to work in
  night shift on par with that of men workers subject to certain
  reasonable terms and conditions particularly regarding their health,
  safety and honour will increase their earning capacity resulting in
  their empowerment.

\item The Central Government has also circulated a model Shops and
  Establishments (Regulation of Employment and Conditions of Service)
  Bill, 2016, which has been finalised after detailed deliberations
  and consultation process, to all the State Governments for
  consideration.

\item In view of the above, the Government of Maharashtra considered
  it expedient to enact a new law, on the lines of the model Bill
  circulated by the Central Government, for regulation of conditions
  of employment and other conditions of service of workers employed in
  various establishments by repealing the existing Maharashtra Shops
  and Establishments Act.

\item The Bill seeks to achieve the above objectives.

\end{sor}


% Memorandum Regarding Delegated Legislation
\begin{mrdl}
  \begin{dllist}
  \item 1 (4) Under this clause, power is taken to the State Government
    to bring the Act into force, by notification in the \emph{Official
      Gazette}, appoint.

  \item 2  Under this clause,—
    \begin{subclause}
    \item under sub-clause (4), power is taken to the State Government
      to declare any other establishment as an establishment for the
      purposes of the Act, by notification in the \emph{Official
        Gazette}.
    \item under sub-clause (10), power is taken to the State Government
      to declare any authority as local authority for the purposes of
      the Act, by notification in the \emph{Official Gazette}.
    \end{subclause}
    \item 5.— Under this clause, power is taken to the State
Government to suspend the operation of all or any of the provisions of
the Act for such period and conditions on account of holidays or occasion,
by notification in the Official Gazette.
    \item 6.— Under this clause,—
      \begin{subclause}
      \item under sub-clause (1), power is taken to the State Government
to prescribe the form of application for registration of
establishments, fees therefor and self-declaration and self-certified
documents to be submitted along with application;
\item under sub-clause (2), power is taken to the State
Government to prescribe the manner of registration of
establishment, form of registration certificate and time within which
the registration certificate is to be granted;
\item under sub-clause (3), power is taken to the State Government
to prescribe the fee and form of renewed registration certificate.
\end{subclause}
\item 7.— Under this clause,—
\begin{subclause}
\item under sub-clause (1), power is taken to the State Government
to prescribe the form of application for intimation of establishments
and self-declaration and self-certified documents to be submitted
along with application and form of register of such intimation;
\item under sub-clause (2), power is taken to the State
Government to prescribe the manner and form in which the employer
shall notify to the Facilitator closing of establishment.
\end{subclause}
\item 9.— Under this clause, power is taken to the State
Government to prescribe the form and the period within which
intimation of change in any of the particular contained in registration
application is to be notified to the Facilitator and fees therefor.



    
  \end{dllist}

\end{mrdl}

\attachpdf{Unit Tests }{unittests.pdf}


\end{document}



%%% Local Variables:
%%% mode: latex
%%% TeX-master: t
%%% End:
