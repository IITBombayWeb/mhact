%\documentclass[default]{mhact}
\documentclass[reprint]{mhact}
\begin{document}


    \actnamealias{2016: Mah. 16}{Maharashtra Act No 24 of 1961}

    \amendlastdate{As modified upto the 11th May 2018}
    \reprintYear{2018}
    \reprintPrice{157}
    
     \printRNInumber{MAHBIL/2009/40123}
      \printVarshSerial{२ }
      \printAnk{२१}
      \printAnkSerial{६}
      \printDiwas{सोमवार}
      \printTarikh{डिसेंबर ५, २०१६}
      \printMarathiTarikh{अग्रहायण १४, १९३७}
      \printPrice{३६}
      \printASKramank{४१}
      \printIntimation{}
    
      \shorttitle{Maharashtra Public Universities Act, 1960}
      \introducer{}
      \introdesig{}
      \psmla{}
    
      \coveringNoteSlip{
        \begin{center}
          \textbf{ MAHARASHTRA LEGISLATURE SECRETARIAT}
        \end{center}

        The following report of the Joint Committee on the Bill to
        provide for academic autonomy and excellence, adequate
        representation through democratic process, transformation,
        strengthening and regulating higher education and for matters
        connected therewith or incidental thereto which was presented
        to the Maharashtra Legislative Assembly on the 5th December,
        2016 is, in accordance with the provisions of sub-rule (4) of
        Rule 129 of the Maharashtra Legislative Assembly Rules,
        published for general information :—
      }

      \originDepartment{
        \begin{center}
          \textbf{DEPARTMENT}\\
Mantralaya, Madam Cama Marg, Hutatma Rajguru Chowk,\\
Mumbai 400 032, dated the 10th March 2019.
        \end{center}
     }
  


    \longtitle{A Bill to provide for academic autonomy and excellence, adequate representation through democratic process, transformation, strengthening and regulating higher education and for matters connected therewith or incidental thereto.}
    

    \workflowstep{As amended by the Joint Committee}

\prologue{%
      \begin{center}
        \textbf{
        Composition of the Joint Committee of Both the Houses on
        L.A. Bill No. XVI of 2016 - The Maharashtra Public
        Universities Bill, 2016}
      \end{center}

      Members:
      \begin{enumerate}
      \item Shri Vinod Tawade, The Hon'ble Minister for Higher and Technical
        Education
      \item Shri Patangrao Kadam, M.L.A.
      \item Shri Dilip Walse-Patil, M.L.A.
      \item Shri Rajesh Tope, M.L.A.
      \item Shri D.P.Sawant, M.L.A.
      \item Shri Rajendra Patani, M.L.A.
      \item Shri Virendra Jagtap, M.L.A.
      \item Adv. Parag Alvani, M.L.A.
      \item Adv. Ashish Shelar, M.L.A.
      \item Dr. Ashok Uike, M.L.A.
      \item Prof. Sangeeta Thombre, M.L.A.
      \item Prof. Medha Kulkarni, M.L.A.
      \item Shri Rahul Patil, M.L.A.
      \item Shri Hemant Patil, M.L.A.
      \item Shri Prakash Abitkar, M.L.A.
      \item Shri Pankaj Bhoir, M.L.A.
      \item Adv. Anil Parab, M.L.C.
      \item Shri Sharad Ranpise, M.L.C.
      \item Shri Satish Chavan, M.L.C.
      \item Shri Kapil Patil, M.L.C.
      \item Prof. Anil Sole, M.L.C.

      \end{enumerate}

      \begin{center}
        \textbf{Report of the Joint Committee}
      \end{center}
      I, the Chairman of the Joint Committee to which L. A. Bill
      No. XVI of 2016 - The Maharashtra Public Universities Bill, 2016
      was referred, having been authorized by the Committee to submit
      the Report on their behalf, present this Report alongwith the
      Bill as amended by the Committee annexed thereto.

      This Bill was introduced in the Legislative Assembly on 5th
      April, 2016. The motion for referring the said Bill to a Joint
      Committee of both the Houses, after obtaining concurrence of the
      Legislative Council was adopted by the Legislative Assembly on
      12th April, 2016.

      The Committee held in all 9 (Nine) meetings. The first meeting
      of the committee was held on 19th May, 2016 to finalise certain
      preliminary matters. The Committee has taken a decision to
      invite amendment/suggestions by the Members of both the Houses
      and people in the said meeting. The opinions of experts such as,
      (1) Shri. Ekbote, Chairman, Progressive Education Society, Pune
      (2) Shri. Sagar Bhalerao, Chairman, Chhatra Bharti, Pune (3)
      Shri.D.N.Dhangare, Former Vice Chancellor, Shivaji University.
      (4) Shri.Dilip Patil, College Librarian Organisation of
      Maharashtra, Nagpur. (5) Prof. S.V. Lavande, General
      Secretary, Maharashtra Federation Of University And College
      Teachers Organisation. (6) Shri.Dinesh Kamble, Maharashtra
      Universities Officers Forum. (7) Shri.Madhu Paranjape,
      Secretary, Bombay University and College Teachers Union, (8)
      Shri. Laxman Shinde, Ex-Senate Member, Dr. Babasaheb Ambedkar,
      Marathwada University, Aurangabad, (9) Smt. Hemlata More,
      General Secretary, PUTA, Pune University Team, Pune, (10)
      Dr.Wagh were heard by the Committee on 8th July, 2016.

      In the meeting held on 29th and 30th June, 2016, 8th and 9th
      July, 2016, 29th and 30th September, 2016 and 20th October,
      2016, the Committee considered the Bill clause by clause and
      finalised the amendments to be made in this Bill. The Committee
      adopted this Report with amendments in its meeting held on 16th,
      November, 2016

      \begin{center}
        \textbf{Report of Joint Committee on L.A. Bill No. XVI of 2016
          Maharashtra Public Universities Bill, 2016}
      \end{center}
      The observations of the Committee with regard to the amendments
      to be made in various clauses of the Bill are elabourated in the
      following paragraph -

      Clause 2.– This clause contains definitions of various terms
      given in the Bill.

      The definition of ``Collegium of University Teachers'' given in
      Sub-clause(19) provides for, `an electoral college consisting
      of fulltime teachers from University Departments, University,
      Institutions and Conducted Colleges who shall elect from amongst
      themselves as members to the different authorities;'

      The Committee felt that Government's recognition is necessary
      for the teachers getting elected on various authorities of the
      University. Since the said definition does not contain a clear
      mention of the said necessity, the said definition be
      amended. Accordingly, the said sub-clause has been amended to
      provide a definition that, `an electoral college consisting of
      fulltime teachers from University Departments, University
      Institutions and Conducted Colleges, appointed by University,
      who shall elect from amongst themselves as members to the
      different authorities'.

      Clause 5.–This clause provides for the powers and duties of
      Universities:

      The sub-clause 53 provides that, the Universities shall `take
      over, in the public interest, the management of an affiliated
      college, institution or autonomous college or empowered
      autonomous college or cluster of institutions in case where
      irregularities or commissions or omissions of criminal nature by
      the management of such college or institution or mismanagement
      of such college or institution are prima facie evident to the
      committee of enquiry appointed by the University'.

      The Committee felt that, the colleges and recognized
      institutions which are going to be closed with the permission of
      State Government can not be taken over by the
      University. Similarly, since the infrastructure set up on such
      colleges from Government's grant are set up on the lands owned
      by the management of Universities, they can not be taken into
      possession by the Government. Hence the Committee proposed an
      amendment in the said clause that damages may be recovered from
      management of such colleges and it may be made binding upon
      them to pay the same.

      Accordingly, a new sub-clause (82) be inserted after sub-clause
      (81) and the Universities have been accorded rights to conduct
      academic audit from time to time
      

\noindent\begin{tabular}{*{2}{p{0.5\linewidth}}}
         & \\
Mumbai & \multicolumn{1}{r}{Shri. Vinod Tawde,} \\
Dated 16th November 2016 & \multicolumn{1}{r}{Chairman} \\
\end{tabular}
      
} % prologue      
    


    \startact 


    
    \enactingtext{
WHEREAS it is expedient to provide for academic autonomy to non-agricultural and non-medical universities in the State of Maharashtra and to make better provisions therefore;

\del{AND WHEREAS the Government of Maharashtra had appointed committees
under the Chairmanships of Dr.Nigvekar, Dr.Kakodkar, Dr.Takwale and
Mrs. Bansal, with a view to consider and recommend on different
aspects of higher education and learning and to suggest various
measures to ensure such autonomy;}

\ins{AND WHEREAS the Government of Maharashtra had appointed
  committees under the Chairmanships of Dr. Arun Nigvekar, Dr. Anil
  Kakodkar, Dr.Ram Takwale and Late Mrs.Kumud Bansal with a view to
  consider and recommend on different aspects of higher education and
  learning and to suggest various measures to ensure such autonomy;}

AND WHEREAS after considering the recommendations of the said
committees the Government of Maharashtra considers it expedient to
make a law to provide for academic autonomy and excellence, adequate
representation through democratic process, transformation,
strengthening and regulating higher education and to regulate the
non-agricultural and non-medical universities in the State of
Maharashtra in more effective manner, to provide for participation of
universities in social and educational spheres, to establish
Maharashtra State Commission for Higher Education and Development, to
constitute various Boards, 
\begin{amendins}{These words were inserted by
    \actref{Mah.}{2013}{38}{9c}}
and to repeal the Maharashtra Universities
Act, 1994 \actnote{\actref{Mah.}{1994}{35}{}}
\end{amendins}
%\ins{\actref{Mah.}{1994}{35}{}};
 it is hereby enacted in
the Sixty-Seventh Year of the Republic of India as follows }
    

      \chapter{Preliminary}
      

      \section{Short title and Commencement.}
      

      \begin{subsectionlist}
    

        \lblitem{1} 
  \begin{amendins}{These words were substituted for the words ``when a
      society has passed a resolution to change'' by
      \actref{Mah.}{1986}{20}{99ab}.}
    This Act may be called the Maharashtra Public Universities Act,
    2016.
 \end{amendins} 
    

        \lblitem{2} 
It shall come into force on such date as the State Government may, by notification in the OfficialGazette, appoint.
\amenddel{The words ``something was here'' were deleted by \actref{Mah.}{1993}{25}{92c}}

       \end{subsectionlist}
    
      \section{Definitions}
      
In this Act, unless the context otherwise requires,


      \begin{subsectionlist}
    

        \lblitem{1} 
``academic services unit'' means university science and instrumentation centre, academic staff college, computer centre, university printing press or any other unit providing specialized services for the promotion of any of the objectives of the university ;


        \lblitem{2} 
``adjunct professor'', ``adjunct associate professor'' or ``adjunct assistant professor'' means a person from industry, trade, agriculture, commerce, social, cultural, academic or any other allied field who is so designated during the period of collaboration or association with the university ;


        \lblitem{3} 
``Empowered Autonomous College'' means an autonomous college that has acquired `A' grade with cumulative grade point average of 3.0 and above or an equivalent grade in the assessment made by the National Assessment and Accreditation Council or by any other assessment and accreditation agency recognised by State or Central Government, and is given the status of empowered autonomous college by the university to which it is affiliated and is empowered to grant a joint degree with the affiliating university ;


        \lblitem{3} 
``Empowered Autonomous College'' means an autonomous college that is identified by the university Grants Commission as College with potential for Excellence or College Excellence, which has high level grade as specified by the Government by notification in the \emph{Official Gazette} as has been given the status of Empowered Autonomous College by the Authority under the Statutes, with a power to grant degree of such College jointly with the affiliating University;


        \lblitem{\del{4}} 
\del{``Empowered Autonomous Cluster Institutions'' means a group of autonomous colleges or institutions of the same management or educational society that have acquired `A' grade with cumulative grade point average of 3.0 and above or an equivalent grade in the assessment made by the National Assessment and Accreditation Council (NAAC) or by any other recognized assessment and accreditation agency, and is given the status of Empowered Autonomous Cluster Institutions by the university to which it is affiliated and is empowered to grant a joint degree with the affiliating university;}


        \lblitem{\ins{4}} 
\ins{``Empowered Autonomous Cluster Institutions'' means a group of autonomous Colleges or institutions of the same management or educational society which includes the colleges or institutions, identified by the University Grants Commission as College with potential for excellence or College of excellence, which have high level grade as specified by the Government by notification in the \emph{Official Gazette} as has been given the status of Empowered Autonomous Cluster Institution by the Authority under the Statutes, and is empowrered to grant a joint degree with the affiliating University;}

       \end{subsectionlist}
%\secnumhead{ \ins{3}}{\ins{New ground}}
%\secnumhead{\del{2} \ins{3}}{\del{Foreground} \ins{Background}}
%\sectionedit{3}{\del{Background}}
    
       %\sectionedit{\ins{2}}{\ins{Earlier procedure of recognition of
       %    institutions}}
      % \sectionedit{\subs{2}{3}}{\del{Procedure}\ins{Method} of recognition of institutions}
      

      \begin{subsectionlist}
    

        \lblitem{1} 
The management of an institution actively engaged in conducting research or specialized studies for a period of not less than five years, and seeking recognition shall apply to the Registrar of the university with full information regarding the following matters, namely

       \end{subsectionlist}
    
      \section{Procedure for recognition of private education provider}
      

      \begin{subsectionlist}
    

        \lblitem{1} The management of private skills education
        provider seeking recognition from the university to various
        degree, diploma, advanced diploma and certificate courses as
        prescribed by the University as per the National, State level
        policy regarding skill qualification and education framework
        and to the experts engaged for conducting such courses shall
        apply to the Registrar in the prescribed format, with full
        information on the programmes run by the private skills
        education provider and other data as sought in the format,
        before the last day of September of the year preceding the
        year from which the recognition is sought.


        \lblitem{2} 
The Pro-Vice-Chancellor shall communicate the decision of the Vice-Chancellor to the management, on or before the 31st May of the year, in which the management desires to seek recognition.The Pro-Vice-Chancellor shall communicate the decision of the Vice-Chancellor to the management.


        \lblitem{3} 
The time schedule for the process laid down in sub-section (2) to (8) shall be published and completed by the University up to 30th April of the year, in which private skills education provider intends to start various degree, diploma, advance diploma and certificate courses


        \lblitem{\subs{3}{4}} 
The recognition shall be valid for a period of five years. The procedure referred to in sub-sections (1) to (8) shall apply mutatis-mutandis, for continuation of such recognition, from time to time.


        \lblitem{\subs{4}{5}} 
The management desirous of closing down the institution providing
private skills education shall apply to the Registrar on or before the
first day of August of the preceding year, stating fully the grounds
for closure, and pointing out the assets in the form of buildings and
equipment, their original cost, the prevailing market value and the
grants so far received by it either from the UGC, the State Government
or from public funding agencies\actnote{\actref{}{1947}{8}{15a}}

       \end{subsectionlist}
    
\end{document}

