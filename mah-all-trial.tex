\documentclass[gaz8,ordinance]{mhact}
\begin{document}


    \actnamealias{2017: Mah. }{L.C. Bill xxx of 2017}
  
    \shorttitle{Maharashtra Shops and Establishments (Regulation of Employment and Conditions of Service) Bill, 2017 }
    \introducer{}
    \introdesig{}
    \psmla{}
    \introdate{}

\coveringNoteSlip{%
 In pursuance of clause (3) of article 348 of the Constitution of India, the following translation in English of the Maharashtra Village Panchayats and the Maharashtra Zilla Parishads and Panchayat Samitis Amendment Ordinance, 2018 (Mah. Ord. XXI of 2018), is hereby published under the authority of the Governor. 

 \begin{flushright}
   By order and in the name of the Governor of Maharashtra,  \\
   Rajendra G. Bhagwat, \\
   Secretary (Legislation) to Government, \\
   Law and Judiciary Department.
 \end{flushright}

 [Translation in English of the Mumbai Municipal Corporation
 (Amendment) Ordinance, 2019 (Mah.  Ord. XI of 2019), published
 under the authority of the Governor].
}

      \originDepartment{%
        \begin{center}
          
          \textbf{Rural Development Department}
          
          Bandhkam Bhavan, 25 Marzban Path \\
          Fort, Mumbai 400001, dated the 11th October 2018
        \end{center}
}

    
    \startact 

    \longtitle{ A Bill to provide for the regulation of conditions of
      employment and other conditions of service of workers employed
      in shops, residential hotels, restaurants, eating houses,
      theatres, other places of public amusement or entertainment and
      other establishments and for matters connected therewith or
      incidental thereto. }

    
    \enactingtext{WHEREAS it is expedient to provide for the regulation of employment 
and other conditions of service of workers employed in shops, 
residential hotels, restaurants, eating houses, theatres, other 
places of public amusement or entertainment and other establishments 
and for matters connected therewith or incidental thereto; it is 
hereby enacted in the Sixty-eighth Year of the Republic of India as 
follows}



        \chapter{Preliminary} 
        
        \section{Short title, extent, application and commencement} 
        
      \begin{subsectionlist}
    
    \item This Act may be called the Maharashtra Shops and Establishments (Regulation of Employment and Conditions of Service) Act, 2017.
    \item It extends to the whole of the State of Maharashtra.
    \item The provisions of this Act, except section 7, shall apply to the 
establishments employing ten or more workers and the provisions of section 
7 shall apply to the establishments employing less than ten workers.
    \item It shall come into force on such date as the State Government may by 
notification in the Official Gazette appoint.
       \end{subsectionlist}
    
        \section{Definitions} 
        
      \begin{subsectionlist}
    
    \item "Chief Facilitator" means the Chief Facilitator appointed as such 
under section 28 of this Act; 
    \item "day" means the period of twenty-four hours beginning at midnight;
    \item “employer” means a person owning or having ultimate control over the affairs of an estab- 
lishment, and includes,–
      \begin{clause}
    
    \item in the case of a firm or association of individuals, a partner or members of the firm 
or association;
    \item in the case of a company, a director of the company;
    \item in the case of an establishment owned or controlled by the Central Government or a 
State Government or any local authority, the person or persons appointed to manage 
the affairs of such establishment by the Central Government or the State Government 
or the local authority, as the case may be;
       \end{clause}
    
    \item “establishment” means an establishment which carries on, any business, trade, manufacture or any journalistic or printing work, or business of banking, insurance, stocks and 
shares, brokerage or produce exchange or profession or any work in connection with, or 
incidental or ancillary to, any business, trade or profession or manufacture; and includes 
establishment of any medical practitioner (including hospital, dispensary, clinic, polyclinic, 
maternity home and such others), architect, engineer, accountant, tax consultant or any 
other technical or professional consultant; and also includes a society registered under the 
Societies Registration Act 1860 
    \actref{}{1860}{63}{}
   , and a charitable or other trust, whether registered or not, 
which carries on, whether for purposes of gain or not, any business, trade or profession or 
work in connection with or incidental or ancillary thereto; and includes shop, residential 
hotel, restaurant, eating house, theatre or other place of public amusement or entertainment; to whom the provisions of the Factories Act, 1948 
    \actref{}{1984}{41}{}
   does not apply ; and includes such 
other establishment as the State Government may, by notification in the Official Gazette, 
declare to be an establishment for the purposes of this Act;
       \end{subsectionlist}
    
        \section{Act not to apply to certain establishments and persons} 
        
      \begin{subsectionlist}
    
    \item Establishments of the Central and State Government;
    \item Establishments pertaining to any kind of educational activities 
(excepting those where coaching or tuition classes are conducted by 
individual persons or any institutions other than those,---
      \begin{clause}
    
    \item affiliated to any university established by law, or
    \item recognised by the Divisional Boards under the Maharashtra 
Secondary and Higher Secondary Education Boards Act, 1965 
    \actref{Mah}{1965}{41}{}
   or
    \item recognised by the Directorate of Education or the 
Directorate of Technical Education as a private secondary or 
technical high school, Industrial Training Institute (I.T.I.), 
Polytechnic, Engineering Colleges or other technical institutions 
conducting courses recognised by Government) ;
       \end{clause}
    
       \end{subsectionlist}
    
        \section{Application of Act to other establishments and workers} 
        
      \begin{subsectionlist}
    
    \item Notwithstanding anything contained in this Act, the State 
Government may, by notification in the Official Gazette, declare any 
establishment or class of establishments to which, or any worker or person 
or class of workers or persons to whom, this Act or any of the provisions 
thereof does not for the time being apply, to be an establishment or class of 
establishments or a worker or a person or class of workers or persons to 
which or whom this Act or any provisions thereof with such modifications or 
adaptations as may in the opinion of the State Government be necessary 
shall apply from such date as may be specified in the notification.
    \item On such declaration under sub-section (1), any such establishment 
or class of establishments or such worker or person or class of workers or persons shall be deemed to be an establishment or class of establishments to 
which, or to be a worker or a person or class of workers or persons to whom, 
this Act, applies and all or any of the provisions of this Act with such 
modification or adaptation as may be specified in such declaration, shall apply 
to such establishment or class of establishments or to such worker or persons 
or class of workers or persons.

       \end{subsectionlist}
    
        \chapter{Registration of Establishments} 
        
        \section{Registration of Establishments} 
        
      \begin{subsectionlist}
    
    \item Within a period of 
   \textbf{sixty}
   days from the date of commencement of this Act or the date on which 
   \emph{establishment}
   commences its business, the employer of every establishment employing ten or more workers shall submit application online in a prescribed form for registration to the Facilitator of the local area concerned, together with such fees and such self-declaration and self-certified documents as may be prescribed, containing---

      \begin{clause}
    
    \item the name of the employer and the manager, if any;
    \item the postal address of the establishment;
    \item such other particulars as may be prescribed :
       \end{clause}
    
       \end{subsectionlist}
    
                \attachpdf[landscape]{Finance Bill}{sample-financebill.pdf}
            \begin{sor}

          \item{The Maharashtra Shops and Establishments Act (LXXIX of 1948) is 
enacted to consolidate and amend the law relating to the regulation of 
conditions of work and employment in shops, commercial establishments, 
residential hotels, restaurants, eating houses, theatres, other places of public 
amusement or entertainment and other establishments.}
          
          \item{The recent information and technology have revolutionized the mode 
of trading whereby it is possible to sell goods and services online without 
any physical, and geographical limitations and time limitations being available 
for twenty-four hours. Therefore, the provisions of the said Act of keeping a 
shop or establishment closed for a business on one day of the week and to 
restrict the opening and closing hours of establishments have become obsolete. 
It has become necessary to provide even platform for offline business to 
compete with online business and to permit to operate shops and 
establishments for twenty-four hours and all days in a week. It will help 
employment generation at large and to increase Gross Domestic Product.}
          
          \item{In view of the above, the Government of Maharashtra considered it 
expedient to enact a new law, on the lines of the model Bill circulated by the 
Central Government, for regulation of conditions of employment and other 
conditions of service of workers employed in various establishments by 
repealing the existing Maharashtra Shops and Establishments Act.}
          
          \item{The Bill seeks to achieve the above objectives.}
          
          \item{}
          The Maharashtra Shops and Establishments Act (LXXIX of 1948) is 
enacted to consolidate and amend the law relating to the regulation of 
conditions of work and employment in shops, commercial establishments, 
residential hotels, restaurants, eating houses, theatres, other places of public 
amusement or entertainment and other establishments.The recent information and technology have revolutionized the mode 
of trading whereby it is possible to sell goods and services online without 
any physical, and geographical limitations and time limitations being available 
for twenty-four hours. Therefore, the provisions of the said Act of keeping a 
shop or establishment closed for a business on one day of the week and to 
restrict the opening and closing hours of establishments have become obsolete. 
It has become necessary to provide even platform for offline business to 
compete with online business and to permit to operate shops and 
establishments for twenty-four hours and all days in a week. It will help 
employment generation at large and to increase Gross Domestic Product.In view of the above, the Government of Maharashtra considered it 
expedient to enact a new law, on the lines of the model Bill circulated by the 
Central Government, for regulation of conditions of employment and other 
conditions of service of workers employed in various establishments by 
repealing the existing Maharashtra Shops and Establishments Act.The Bill seeks to achieve the above objectives.

\sorSignature{%
\begin{tabular}{*{2}{p{0.5\linewidth}}}
        Mumbai & 
            \multicolumn{0}{r}{CH . VIDYASAGAR RAO,}
            \\
Dated 24th February 2019 & 
            \multicolumn{0}{r}{Governor of Maharashtra}
\end{tabular}

        \begin{flushright}
By order and in the name of the Governor of Maharashtra,
        \end{flushright}
        

        \begin{flushright}
           
MANISHA PATANKAR-MHAISKAR, \\
Principal Secretary to Government.
        \end{flushright}
} 
\end{sor}

\begin{mrdl}

  \intro{The Bill involves the following proposals for delegation of legislative
    powers, namely:---}
  
\begin{dllist}

        \item{1 (4).— Under this clause, power is taken to the State 
Government to bring the Act into force, by notification in the 
   \emph{Official
Gazette}
  , appoint.}
      
        \item{2. Under this clause,—

\begin{enumerate}[label=(\arabic*)]
      \item under sub-clause (4), power is taken to the State Government
      \item under sub-clause (10), power is taken to the State
\end{enumerate}
under sub-clause (4), power is taken to the State Government 
	to declare any other establishment as an establishment for the 
	purposes of the Act, by notification in the 
   \emph{Official Gazette}
  .under sub-clause (10), power is taken to the State 
	Government to declare any authority as local authority for the 
	purposes of the Act, by notification in the 
   \emph{Official Gazette}
  .}
      
        \item{4.— Under this clause, power is taken to the State 
Government to declare by notification in the Official Gazette that the 
Act or any provisions thereof shall apply to any establishment or class 
of establishments to which or any worker or class of workers to whom 
the said Act does not apply.}
      \end{dllist}

\mrdlSignature{%


\begin{tabular}{*{2}{p{0.5\linewidth}}}
Vidhan Bhavan : & Dr. ANANT KALSE,\\
Mumbai, & \multicolumn{1}{r}{Principal Secretary,}\\
Dated the 8th August 2017. & Maharashtra Legislative Assembly.\\
\end{tabular}
}

\end{mrdl}

                \attachpdf{Schedule Some Table}{unittests.pdf}
             
\end{document}

