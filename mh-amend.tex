\documentclass[gaz5a]{mhact}
\begin{document}


    \actnamealias{2016: Mah. 16}{L.A. Bill 16 of 2019}
  
      \printRNInumber{MAHBIL/2009/40123}
      \printVarshSerial{२ }
      \printAnk{२१}
      \printAnkSerial{६}
      \printDiwas{सोमवार}
      \printTarikh{डिसेंबर ५, २०१६}
      \printMarathiTarikh{अग्रहायण १४, १९३७}
      \printPrice{३६}
      \printASKramank{४१}
      \printIntimation{}
    
      \shorttitle{Maharashtra Public Universities Act,}
      \introducer{}
      \introdesig{}
      \psmla{}
      \introdate{}
    


      \originDepartment{
        \begin{center}
          \textbf{DEPARTMENT}\\
Mantralaya, Madam Cama Marg, Hutatma Rajguru Chowk,\\
Mumbai 400 032, dated the 10th March 2019.
        \end{center}
     }
  


    \longtitle{A Bill to provide for academic autonomy and excellence, adequate representation through democratic process, transformation, strengthening and regulating higher education and for matters connected therewith or incidental thereto.}
    


    \startact 

    \enactingtext{
WHEREAS it is expedient to provide for academic autonomy to non-agricultural and non-medical universities in the State of Maharashtra and to make better provisions therefore;

\del{AND WHEREAS the Government of Maharashtra had appointed
  committees under the Chairmanships of Dr.Nigvekar, Dr.Kakodkar,
  Dr.Takwale and Mrs. Bansal, with a view to consider and recommend on
  different aspects of higher education and learning and to suggest
  various measures to ensure such autonomy;}

\ins{AND WHEREAS the Government of Maharashtra had appointed
  committees under the Chairmanships of Dr. Arun Nigvekar, Dr. Anil
  Kakodkar, Dr. Ram Takwale and Late Mrs. Kumud Bansal with a view to
  consider and recommend on different aspects of higher education and
  learning and to suggest various measures to ensure such autonomy;}

AND WHEREAS after considering the recommendations of the said committees the Government of Maharashtra considers it expedient to make a law to provide for academic autonomy and excellence, adequate representation through democratic process, transformation, strengthening and regulating higher education and to regulate the non-agricultural and non-medical universities in the State of Maharashtra in more effective manner, to provide for participation of universities in social and educational spheres, to establish Maharashtra State Commission for Higher Education and Development, to constitute various Boards, and to repeal the Maharashtra Universities Act, 1994; it is hereby enacted in the Sixty-Seventh Year of the Republic of India as follows
}
    

      \chapter{Preliminary}
      

      \section{Short title and Commencement.}
      

      \begin{subsectionlist}
    

        \lblitem{1}
This Act may be called the Maharashtra Public Universities Act, 2016.

      

        \lblitem{2}
It shall come into force on such date as the State Government may, by notification in the OfficialGazette, appoint.

      
       \end{subsectionlist}
    
      \section{Definitions}
      

      \begin{subsectionlist}
    

        \lblitem{1}
academic services unit means university science and instrumentation centre, academic staff college, computer centre, university printing press or any other unit providing specialized services for the promotion of any of the objectives of the university ;

      

        \lblitem{2}

``adjunct professor'', ``adjunct associate professor'' or ``adjunct assistant professor'' means a person from industry, trade, agriculture, commerce, social, cultural, academic or any other allied field who is so designated during the period of collaboration or association with the university ;

      

        \lblitem{3}

``Empowered Autonomous College'' means an autonomous college that has acquired 'A' grade with cumulative grade point average of 3.0 and above or an equivalent grade in the assessment made by the National Assessment and Accreditation Council or by any other assessment and accreditation agency recognised by State or Central Government, and is given the status of empowered autonomous college by the university to which it is affiliated and is empowered to grant a joint degree with the affiliating university ;

      

        \lblitem{4}
``Empowered Autonomous Cluster Institutions'' means a group of autonomous colleges or institutions of the same management or educational society that have acquired 'A' grade with cumulative grade point average of 3.0 and above or an equivalent grade in the assessment made by the National Assessment and Accreditation Council (NAAC) or by any other recognized assessment and accreditation agency, and is given the status of Empowered Autonomous Cluster Institutions by the university to which it is affiliated and is empowered to grant a joint degree with the affiliating university;

      
       \end{subsectionlist}
    
      \section{Procedure of recognition of institutions}
      

      \begin{subsectionlist}
    

        \lblitem{1}
The management of an institution actively engaged in conduct- ing research or specialized studies for a period of not less than five years, and seeking recognition shall apply to the Registrar of the university with full information regarding the following matters, namely

      
       \end{subsectionlist}
    
      \section{Procedure for recognition of private education provider}
      

      \begin{subsectionlist}
    

        \lblitem{1}
The management of private skills education provider seeking rec- ognition from the university to various degree, diploma, advanced diploma and certifi- cate courses as prescribed by the University as per the National, State level policy regarding skill qualification and education framework and to the experts engaged for conducting such courses shall apply to the Registrar in the prescribed format, with full information on the programmes run by the private skills education provider and other data as sought in the format, before the last day of September of the year preceding the year from which the recognition is sought.
      

        \lblitem{2}
The Pro-Vice-Chancellor shall communicate the decision of the Vice-Chan- cellor to the management, on or before the 31st May of the year, in which the manage- ment desires to seek recognition.
      

        \lblitem{3}
The recognition shall be valid for a period of five years. The procedure referred to in sub-sections (1) to (8) shall apply mutatis-mutandis, for continuation of such recognition, from time to time.
      

        \lblitem{4}
The management desirous of closing down the institution providing pri- vate skills education shall apply to the Registrar on or before the first day of August of the preceding year, stating fully the grounds for closure, and pointing out the assets in the form of buildings and equipment, their original cost, the prevailing market value and the grants so far received by it either from the UGC, the State Government or from public funding agencies

      
       \end{subsectionlist}
    
\end{document}

